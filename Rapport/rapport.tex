%%% Local Variables:
%%% TeX-parse-self: t
%%% TeX-auto-save: nil
%%% End:

\documentclass{article}

\title{Projet MiRitH}
\author{Maxime Coute \and Jules Magois}

\begin{document}
\maketitle

\section{Proposition de structure du code}

\begin{enumerate}
  \item un fichier \verb!matrix.c! pour gérer les opérations sur les matrices:
        \begin{itemize}
          \item allocation de mémoire
          \item libération de mémoire
          \item addition de deux (ou une liste de) matrices
          \item multiplication de deux matrices
        \end{itemize}
  \item un fichier \verb!key_generation.c! pour générer la clé,
  \item un fichier \verb!party.c! qui implémente les calculs de chaque partie,
  \item un fichier \verb!main.c! qui implémente la signature.
\end{enumerate}

\section{Bibliothèques utilisées}
\begin{enumerate}
  \item \verb!gmp! pour la génération de nombres aléatoires, et les calculs sur les grands entiers,
    \item \verb!openssl! pour l'utilisation du hash \emph{Keccak}.
\end{enumerate}

\end{document}
