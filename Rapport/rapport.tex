%%% Local Variables:
%%% TeX-parse-self: t
%%% TeX-auto-save: nil
%%% End:

\documentclass{article}
\usepackage[french]{babel}

\title{Projet MiRitH}
\author{Maxime Coute \and Jules Magois}

\begin{document}
\maketitle

\section{Explication générale de la signature}
\subsection{Problème MinRank}
\subsection{Authentification zero-knowledge utilisant MPC-in-the-head}
\subsection{Fiat-Shamir pour obtenir une signature}
\subsection{Sécurité}

Soient:
\begin{enumerate}
  \item \(N\) le nombre total de parties,
  \item \(\tau\) le nombre de tours effectués.
\end{enumerate}

La probabilité qu'un faux signataire arrive à une signature correcte est
proportionnelle à \(N^{-\tau}\).

La taille d'une signature (en nombre de bits) est proportionnelle à  \(\tau\).

On peut donc faire différents choix de \(N\) et \(\tau\) en fonction de nos besoins:
\begin{itemize}
  \item \(N\) petit et \(\tau\) élevé pour générer rapidement une signature de grande taille,
  \item \(N\) grand et \(\tau\) faible pour générer une signature courte.

\section{Proposition de structure du code}

\begin{enumerate}
  \item un fichier \verb!field_arithmetics.c! pour l'addition et la multiplication
        sur \(GF(16)\),
  \item un fichier \verb!constants.c! contenant les constantes utiles pour tous les autres fichiers:
        \begin{itemize}
          \item la définition du corps \verb!GF_16!
          \item quelques paramètres de signature standards
          \item les tables d'addition et de multiplication dans \verb!GF_16!
       \end{itemize}
  \item un fichier \verb!matrix.c! pour gérer les opérations sur les matrices:
        \begin{itemize}
          \item allocation de mémoire
          \item libération de mémoire
          \item addition de deux (ou une liste de) matrices
          \item multiplication de deux matrices
        \end{itemize}
  \item un fichier \verb!key_generation.c! pour générer la clé,
  \item un fichier \verb!party.c! qui implémente les calculs de chaque partie,
  \item un fichier \verb!main.c! qui implémente la signature.
\end{enumerate}

\section{Bibliothèques utilisées}
\begin{enumerate}
  \item \verb!gmp! pour la génération de nombres aléatoires,
  \item \verb!openssl! pour l'utilisation du hash \emph{Keccak}.
\end{enumerate}

\end{document}
